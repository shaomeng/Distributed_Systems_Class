File systems are a crucial component of many large-scale systems, 
namely the supercomputers and internet-based service providers.
%
While the computational capacity of modern supercomputers are keep growing,
the file system performance is not keep up the pace, in terms of both
the storage capacity and data transfer throughput.
%
Distributed file systems are widely used to tackle this problem.
%
On the one hand, distributed file systems enables a large number of commodity
storage devices to connect together and act like a unified storage space, 
which expands the storage capacity.
%
On the other hand, distributed file systems can potentially aggregate file
I/O operations from single storage devices together, providing a high
data transfer throughput. 
%
This survey paper sheds some light on the architecture as well as various
performance issues of distributed file systems.

%This survey paper identifies common performance issues in 
%the distributed file systems, 
%and explores approaches to minimize impacts of these issues.
%
This survey paper specifically looks into three popular distributed 
file systems: Google File System GPFS, and Lustre.
%
Google File System~\cite{ghemawat2003google} is designed by Google and used 
exclusively by Google.
%
GPFS~\cite{Schmuck2002,barkes1998gpfs} is a proprietary system owned by IBM, and
it also powers many supercomputers.
%
Lustre~\cite{Schwan2003} is open-sourced and powers many of the 
world's fastest supercomputers.
%
We will especially focused on three aspects of these file systems:
1) different architecture design,
2) file partitioning scheme, and 3) efforts and demonstrations to achieve
higher performance using these three file systems.
