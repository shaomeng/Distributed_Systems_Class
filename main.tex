\documentclass[12pt,twocolumn]{IEEEtran11}

\usepackage{times}
\usepackage{epsfig}
\usepackage[T1]{fontenc}
\usepackage{graphicx}
\usepackage{subfigure}
\def\BibTeX{{\rm B\kern-.05em{\sc i\kern-.025em b}\kern-.08em
    T\kern-.1667em\lower.7ex\hbox{E}\kern-.125emX}}

\oddsidemargin -15pt
\evensidemargin -15pt
\leftmargin 0 pt
\topmargin -30pt
\textwidth = 6.9 in
\textheight = 9.0 in

\newcommand{\itembase}{\setlength{\itemsep}{0pt}}
\newcommand{\eg}{{\it e.g., }}
\newcommand{\ie}{{\it i.e., }}
\graphicspath{{FIG/}}

\begin{document}
\bibliographystyle{IEEE}

\title{\Large \bf 
Survey on Bottlenecks and Best Practices of Three Implementations of
Distributed File Systems
}
\author{
Samuel Li\\
Information and Computer Science Department\\
University of Oregon\\
{\em samuelli@cs.uoregon.edu}
}
\maketitle
% You have to do this to suppress page numbers.  Don't ask.
%\pagestyle{empty}
\begin{abstract}
File systems are a crutial component of supercomputers.
%
Often, file system of a supercomputer consists of many file servers
to provide capacity, reliability, redundency, and consistency
required by a supercomputer.
%
These file servers are organized as a distributed system
to provide functionalities of a file system.
%
A distributed file system can have several performance challenges,
due to the distributed nature of the file system itself,
and the high concurrency nature of the supercomputer system.
%
This survey paper studies performance issues of two most widely 
used file systems: Lustre and GPFS.
%
The scope of performance issues include identificatino of bottlenecks
and best approaches to minimize impact of these bottlenecks.
\end{abstract}

%\begin{keywords} 
%Quality Adaptive Streaming, Peer-to-Peer, Internet
%\end{keywords}

\section{Introduction}
\label{sec:intro}
File systems are a crucial component of many large-scale systems, 
namely the supercomputers and internet-based service providers.
%
While the computational capacity of modern supercomputers are keep growing,
the file system performance is not keep up the pace, in terms of both
the storage capacity and data transfer throughput.
%
Distributed file systems are widely used to tackle this problem.
%
On the one hand, distributed file systems enables a large number of commodity
storage devices to connect together and act like a unified storage space, 
which expands the storage capacity.
%
On the other hand, distributed file systems can potentially aggregate file
I/O operations from single storage devices together, providing a high
data transfer throughput. 
%
This survey paper sheds some light on the architecture as well as various
performance issues of distributed file systems.

%This survey paper identifies common performance issues in 
%the distributed file systems, 
%and explores approaches to minimize impacts of these issues.
%
This survey paper specifically looks into three popular distributed 
file systems: Google File System GPFS, and Lustre.
%
Google File System~\cite{ghemawat2003google} is designed by Google and used 
exclusively by Google.
%
GPFS~\cite{Schmuck2002,barkes1998gpfs} is a proprietary system owned by IBM, and
it also powers many supercomputers.
%
Lustre~\cite{Schwan2003} is open-sourced and powers many of the 
world's fastest supercomputers.
%
We will especially focused on three aspects of these file systems:
1) different architecture design,
2) file partitioning scheme, and 3) efforts and demonstrations to achieve
higher performance using these three file systems.


\section{Brief on Surveyed Papers}
\label{sec:brief}

The following papers characterize behaviors of distributed file systems
on supercomputers:
\cite{Xie2012}, \cite{Henschel2012}, \cite{Crosby2009}, \cite{Borrill2009}.

The following papers discuss approaches to better work with distributed 
file systems given the characters above:
\cite{Shipman2010}, \cite{Yu2006}, \cite{Tian2011}, \cite{Lofstead2010},
\cite{Lofstead2009}.


%% file citations.bib contains all the biblography
\bibliography{main}
\end{document}
